\chapter{Introducción}
\towrite[caption={Introducción}]{
Pequeña presentación resumen sobre el TFG y su objetivo principal
}
\section{Definición del Problema}
\towrite[caption={Definción del Problema}]{
Partir de cómo se hacen los análisis médicos, los pasos 
para segmentar y generar el modelo 3D sobre el cúal se hace alguna inferencia 
médica 
\par
(Añadir imágenes de ejemplo, su segmentación y una reconstrución).
\par 
Comentar algunos softwares de visualización y generación de volúmenes.
\par 
Hablar de las posibles distorsiones que pueden ocurrir 
y como puede influir en el predicado médico 
(Quizás añadir algún ejemplo visual de una distorsión que ocasione confusión). 
\par 
Citar superficialmente disciplinas y métodos que se intentan aplicar 
para resolver este problema.
}
\section{Motivación}
\towrite[caption={Escribir Motivación}]{
  Mencionar el avance de la IA, de los métodos IQA y la problemática de 
  de dar al médico la mejor calidad posible. 
  \par 
  Hablar de la posibilidad de utilizar la métrica para desarrollar mejoras 
  sobre los métodos de compresión y generación de volúmenes. Citar ejemplos 
  de algunos métodos IQA o el caso de éxito de Netflix con su métrica para 
  videos que supuso una gran mejora en el algoritmo de compresión y streaming 
  en red.
  \par    
  Hablar de los altos costes que provoca tener que re-hacer exámenes médicos.
  \par 
  Hablar de los pocos métodos y datasets que existen para este caso especifico 
  de imágenes biomédicas en 3D y sus distorsiones. 
}
\section{Objetivos}
\towrite[caption={Objetivos}]{
  Hablar del esquema secuencial que hemos seguido: 
  \par 
  1. Plantear la resolubilidad con uso de caracteríticas NSS (Natural scene 
  statistics) y modelos de aprendizaje automático como SVR, KNNRegressor, Ridge y Tree Regressor. 
  2. Analizar la necesidad del uso de modelos de aprendizaje profundo. (A juego con lo anterior) 
  \par 
  Con ello, el objetivo principal sería algo del estilo: Contribuir a la 
  creación de una métrica de nivel de calidad de imágenes 3D automática,
  acorde con el HVS (Human vision system), y con alto rendimiento para 
  las distorsiones presentar en este ámbito. Generar, o más bien, demonstrar 
  la posiblidad de generar datasets sintéticos para resolver este problema. 
  Destacando que usaremos uno de partida, el LS-SJTU-PCQA. 
}
\section{Planificación del proyecto}
\towrite[caption={Planificación de Proyecto}]{
    Aquí lo que haría sería un diagrama de Gantt con la planificación 
    desde el mes de febrero. Donde las primeras semanas sería apenas 
    de lectura de artículos y evaluación del estado de arte y posiblidades.
    \par 
    Luego las siguientes semanas serían algo de estilo de implemetanción 
    del modelo sencillo, implementación del modelo NSS complejo (Descomposición 
    del vecindario de los puntos en valores singulares para calcular componentes 
    como esfericidad, planaridad...), implementación del modelo DL, 
    luego implementación de mejoras sobre ambos. 
    \par 
    Hablaría de los saltos que he hecho entre ML con NSS, a DL a ML NSS,
    las pausas, más lecturas y vueltas a implementar más y otras mejoras.
    \par 
    Por último, pondría el análisis de resultados y comienzo de escritura de 
    este documento. Pensaba utilizar el esquema de agrupación que utilizó 
    Valentino en el diagrama de Gantt.
    \par 
    Para terminar este apartado lo que haría sería un presupuesto suponiendo 
    haber trabajado X horas al día durante los días laborales con un salario Y.
    Calcular las horas, multiplicar y asumir otros gastos aparte como lo son 
    el portátil, colab y otros. 
}
