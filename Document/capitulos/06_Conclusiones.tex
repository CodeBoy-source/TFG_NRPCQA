\chapter{Conclusiones y Trabajos Futuros}
La estimación de la calidad de imágenes es un problema esencial a la hora 
de optimizar el formato y visualización de la información, además de ser 
un problema de suma importancia para el ámbito biomédico. Este TFG aborda 
la obtención de una métrica de estimación de calidad capaz de evaluar volúmenes 
3D, en concreto nubes de puntos, del ámbito biomédico sin referencia para poder asistir 
en la mejora de los algoritmos de reconstrucción y visualización 3D. 

En primer lugar, se realizó un estudio de la literatura relativa a la estimación 
de calidad de imágenes 2D, desde los métodos basado en extracción de características 
de escenas NSS y modelos de ML, hasta la extracción automática con DL. 
De aquí, se pudo observar que hay una tendencia, justificada, a los métodos de DL 
debido a su capacidad de generalización, hecho observado en los resultados obtenidos en diversas 
pruebas realizadas. También se dio a conocer las diferencias en complejidad entre la 
estimación de calidad con referencia, donde se posee la imagen original a comparar con 
la versión distorsionada, frente a los métodos sin referencia, donde la versión 
original no está disponible para la comparación.
A continuación, se estudió el uso de estos métodos sobre imágenes 
médicas 2D, donde se presenta la dificultad de resolución del problema, 
parcialmente por ser el subproblema número 2, sin referencia, y debido a que los 
métodos más conocidos de imágenes genéricas 2D no son directamente aplicados a imágenes 
médicas 2D. También se observa una tendencia a métodos de DL. Posteriormente, 
se estudia el estado del arte de métodos dedicados a representaciones tridimensionales, 
en concreto las nubes de punto que son la representación habitual en el ámbito biomédico. 
Aquí se presenta la complejidad computacional que impone las representaciones 3D y 
la extracción de características sobre ellas. Se observa también la misma tendencia 
que en el ámbito 2D, un salto de ML a DL debido a los resultados obtenidos 
sobre diversos dataset. Por último, se concluye que no existe hasta el momento 
otra investigación que haya tomado el enfoque novedoso de estimar la calidad de 
representaciones biomédicas 3D directamente. 

Se realizó un análisis de los métodos SOTA de estimación de calidad de objetos 
3D, tanto desde la perspectiva de ML como del DL. El primero hace uso de características 
extraídas manualmente utilizando conocimiento humano sobre el HVS, como fenómenos 
de planaridad, esfericidad, anisotropía, curvatura, linealidad y consistencia de 
colores de las nubes de puntos, que luego se utilizan para estimar una regresión por 
SVM. En comparativa, de los modelos ML se utilizó un modelo capaz de extraer 
información estática y dinámica de nubes de puntos haciendo uso de múltiples 
proyecciones 2D y de un vídeo del objeto 3D rotando. De esta forma, podemos 
simular el HVS. Para ambos modelos se proponen ajustes y pequeñas mejoras basadas 
en recientes publicaciones y se comparan los resultados. 
Al igual que la tendencia del estado del arte, sale exitoso el modelo DL. 

Para la validación sobre un conjunto de datos médicos se necesitó el desarrollo 
de un conjunto de datos sintético debido a la no existencia de un conjunto de 
datos públicos para este análisis. Para ello se estudiaron y se fabricaron las distorsiones más 
comunes del ámbito biomédico con respecto a las representaciones 3D. Para las etiquetas, 
se hizo uso de las métricas SOTA del problema con referencia, que es el subproblema 
más avanzado y que en diversas publicaciones se demuestra la alta correlación 
de esas métricas y el HVS, pudiendo así justificarse el uso de esas 
etiquetas generadas antes que la ejecución de un experimento de estimación 
manual para la obtención del MOS según los estándares, experimento que se 
sale del marco temporal y planificación de este proyecto. Se generaron 
un total de 385 representaciones médicas 3D distorsionadas, 11 nubes de puntos 
base, 5 distorsiones a 7 niveles cada una. En las distorsiones se simula 
tanto errores de transmisión, compresión como el movimiento del paciente.

\towrite[resultados]{escribir los resultados aquí.}

Por lo tanto, se concluye que se han completado satisfactoriamente los objetivos 
planteados, determinando la posibilidad de resolución del problema adaptado 
al ámbito biomédico y abriendo puertas a futuras investigaciones. Todo el código
se encuentra disponible en el repositorio de GitHub \url{https://github.com/CodeBoy-source/TFG_NRPCQA},
a excepción de las imágenes médicas ya que son datos confidenciales.

Siendo un proyecto en una nueva línea de investigación, existen varias ampliaciones 
lógicas que se pueden realizar a este proyecto. Por un lado, se podría proponer 
obtener un MOS manual según los estándares y volver a validar los resultados obtenidos
entre los distintos modelos. Así como utilizar ese conjunto de MOS manual sobre imágenes médicas 
para normalizar las etiquetas sintéticas como lo hacen en la publicación original, 
permitiendo de un conjunto pequeño extraído manualmente obtener uno varias 
veces más grande. También, para mejorar el método propuesto se podría permitir 
que los pesos del modelo utilizado para la extracción de características 
del vídeo fueran alteradas. Además, se podría buscar simular las distorsiones 
sobre el conjunto de imágenes 2D generadas tras el examen en vez de hacerlo 
sobre la representación 3D final. 

Por otro lado, se pueden explorar otros métodos que procesen modelos 3D directamente, 
o que hagan uso de proyecciones y de la nube de puntos simultáneamente como en MM-PCQA\cite{MM-PCQA}.
Actualmente, ha crecido el número de publicaciones de adaptaciones de PointNet\cite{PointNet} y 
PointNet++\cite{PointNet++} para resolver distintos problemas de nubes de puntos, 
por lo que quizás se podría adaptar para la resolución de este problema, como 
el método de \cite{ResSCNN}\footnote{Este método no se empleo porque no se 
logró realizar la ejecución del modelo en los entornos disponibles 
del desarrollo del proyecto.} y evitar así 
la inevitable pérdida de información al proyectar en 2D.


