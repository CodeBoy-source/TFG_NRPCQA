\chapter{Conclusiones y Trabajos Futuros}
La estimación de la calidad de imágenes (IQA) es un problema esencial a la hora 
de optimizar el formato y visualización de la información, además es de 
suma importancia para el ámbito biomédico. 
Este TFG aborda la obtención de una métrica de estimación de calidad capaz de evaluar 
representaciones 3D sin referencia, en concreto nubes de puntos, del ámbito biomédico para poder asistir 
en la mejora de los algoritmos de reconstrucción y visualización de dichos objetos. 

En primer lugar, se realizó un estudio de la literatura relativa a la estimación de calidad 
de imágenes 2D, desde los métodos basados en extracción de características de escenas naturales y modelos de ML, 
hasta la extracción automática con DL. 
A continuación, se estudió el uso de estos y otros métodos sobre imágenes médicas 2D.
Posteriormente se analizó el estado del arte de métodos dedicados a representaciones tridimensionales. 
Se observa un salto en complejidad teórica y computacional al tratarse de problemas 
en tres dimensiones. 
Por último, se concluye que no existe hasta el momento 
otra investigación que haya tomado el enfoque novedoso de estimar la calidad de 
reconstrucciones biomédicas 3D. 

Ante la falta de propuestas específicas, el trabajo parte de la implementación 
de métodos relevantes del estado del arte de estimación de calidad de objetos 3D, 
tanto desde la perspectiva de métodos tradicionales (ML, en donde la extracción 
de características y la clasificación son etapas independientes) como de métodos 
\emph{end-to-end} DL. El primero hace uso de características 
extraídas manualmente utilizando conocimiento humano sobre el sistema visual humano (HVS), 
como fenómenos de planaridad, esfericidad, anisotropía, curvatura, linealidad y consistencia de 
colores de las nubes de puntos, que luego se utilizan para estimar una regresión por 
SVM. En cuanto a modelos basados en DL, se utilizó un modelo capaz de extraer 
información estática y dinámica de nubes de puntos haciendo uso de múltiples 
proyecciones 2D y de un vídeo del objeto 3D rotando. De esta forma, podemos  
simular el HVS. En ambos casos se proponen ajustes y pequeñas mejoras basadas 
en recientes publicaciones y se comparan los resultados con la propuesta original. 

Para la validación sobre un conjunto de datos médicos fue necesaria la creación 
de un conjunto de datos sintético debido a la no existencia de un conjunto de 
datos públicos para este análisis. Para ello se estudiaron y se fabricaron las distorsiones más 
comunes del ámbito biomédico con respecto a las representaciones 3D. 
Para evitar la problemática logística del etiquetado a través de la 
evaluación humana sobre el dataset sintético, 
fue necesario estudiar el problema IQA con referencia y hacer uso de las métricas 
más empleadas. Dichas métricas demostraron una alta correlación con el HVS, 
justificando así su uso para generar etiquetas artificiales.
Se generaron un total de 385 representaciones médicas 3D distorsionadas, 11 nubes de puntos 
base, 5 distorsiones a 7 niveles cada una. En las distorsiones se simula 
tanto errores de transmisión, compresión como el movimiento del paciente.

Como primera conclusión de nuestra experimentación base, 
siguiendo la tendencia del estado del arte, el modelo DL sale exitoso en la 
comparativa sobre objetos 3D genéricos. Dicha conclusión es consecuencia 
de lograr replicar satisfactoriamente los resultados de los métodos estudiados 
sobre los conjuntos de datos públicos.
A continuación, se observa que el modelo adaptado de ML (NR3DQA) demuestra no 
ser capaz de determinar con calidad el nivel de distorsiones de las imágenes médicas. 
Sin embargo, el modelo basado en DL (VQA-PC) consigue resultados aceptables con 
una correlación media del 71\%.
Finalmente, se aplican mejoras a los métodos 
y se concluye que, tras entrenar con datos sintéticos de distorsiones similares y 
diversas nubes de puntos, se obtiene una mejora considerable en el modelo basado en DL. 
En concreto, se alcanza una alta correlación (88\%) utilizando la aproximación F2 
(fusión por convolución).

Por lo tanto, se concluye que se han completado satisfactoriamente los objetivos 
planteados, determinando la posibilidad de resolución del problema adaptado 
al ámbito biomédico y abriendo puertas a futuras investigaciones. 
Todo el código se encuentra disponible en el siguiente repositorio de 
GitHub \url{https://github.com/CodeBoy-source/TFG_NRPCQA},
a excepción de las imágenes médicas que son datos confidenciales.

Siendo un proyecto en una nueva línea de investigación, existen varias ampliaciones 
lógicas que se pueden realizar a este proyecto. Por un lado, se podría 
obtener una etiqueta manual con un experimento de evaluación, según los estándares, para 
obtener una opinión media de calidad (MOS) y volver a validar los resultados obtenidos
entre los distintos modelos. Así como utilizar ese conjunto de MOS manual sobre imágenes médicas 
para normalizar las etiquetas sintéticas como lo hacen en la publicación original~\cite{ResSCNN}, 
donde utilizan un conjunto pequeño extraído manualmente para normalizar uno sintético varias 
veces más grande. También, para mejorar el método propuesto, se podría permitir 
que los pesos del modelo utilizado para la extracción de características 
del vídeo fueran alterados en vez de ser solamente un paso previo, de extracción. 
De esta forma se podría guiar el modelo a buscar nuevas características temporales.  
Además, se podría buscar simular las distorsiones 
sobre el conjunto de imágenes 2D generadas tras el examen en vez de hacerlo 
sobre la representación 3D final, teniendo así datos más realistas. 

Por otro lado, se pueden explorar otros métodos que procesen modelos 3D directamente, 
o que hagan uso de proyecciones y de la nube de puntos simultáneamente, como en MM-PCQA~\cite{MM-PCQA}.
Actualmente, ha crecido el número de publicaciones de adaptaciones de PointNet~\cite{PointNet} y 
PointNet++~\cite{PointNet++} para resolver distintos problemas de nubes de puntos, 
por lo que quizás se podría adaptar para la resolución de este problema, como 
el método de ResSCNN~\cite{ResSCNN} y evitar así 
la pérdida de información al proyectar en 2D.
