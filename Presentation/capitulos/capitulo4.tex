\section{Experimentación}
\subsection{Modelo NR3DQA}
\begin{frame}
  \frametitle{Modelo NR3DQA\footnotemark[12]}
\begin{table}[htp]
  \small
  \begin{center}
    \hspace{-.5cm}
    \begin{tabular}[c]{|c|c|c|c|c|}
      \hline
      \rowcolor[HTML]{FFC702}
      \multicolumn{1}{|c|}{\textbf{Dataset}} & 
      \multicolumn{1}{|c|}{\textbf{Modelo}} & 
      \multicolumn{1}{|c|}{\textbf{Escalado}} & 
      \multicolumn{1}{|c|}{\textbf{PLCC}} &
      \multicolumn{1}{|c|}{\textbf{SROCC}} \\
      \hline
      SJTU & SVM & MinMaxScaler & 0.810325 & 0.777403 \\ 
      \hline 
      WPC & SVM & MinMaxScaler & 0.637953 & 0.634853 \\
      \hline
      Biomédico & SVM & RobustScaler & 0.2017 & 0.1776 \\
      \hline
      Biomédico normalizado & KNNRegressor & RobustScaler & 0.2671 & 0.1882  \\
      \hline
      Biomédico en escala 0-5 & DecisionTree & StandardScaler & 0.309176 & 0.196713 \\
      \hline
    \end{tabular}
  \end{center}
  \caption[Resultados de prueba preliminar con NR3DQA.]{
    Resultados de prueba preliminar con NR3DQA\footnotemark[12]. 
  }
  \label{tab:MedicalNR3DQA}
\end{table}
\footnotetext[12]{\cite{NR3DQA}}

\end{frame}

\begin{frame}
  \frametitle{Modificaciones}
  \vspace{-.7cm}
  \begin{table}[htp]
    \small
    \begin{center}
    \caption[Resultado de mejoras sobre el método NR3DQA.]{Resultado de mejoras sobre el método NR3DQA.}
      \begin{tabular}[c]{|c|c|c|c|c|c|c|}
        \hline
        \rowcolor[HTML]{FFC702}
        \textbf{Dataset} & \textbf{Modelo} & \textbf{Escalado} & \textbf{PLCC} & \textbf{SROCC} \\ 
        \hline
        SJTU & SVM & MinMaxScaler & 0.853709 & 0.820057 \\ 
        \hline 
        WPC & SVM & MinMaxScaler & 0.642356 & 0.62917 \\
        \hline 
        Biomédico & SVM & StandardScaler & 0.344601 &  0.170793 \\
        \hline
        Biomédico escalado & DecisionTree & StandardScaler & 0.30025  & 0.182296 \\
        \hline
      \end{tabular}
    \end{center}
    \label{tab:ImprovNR3DQA}
  \end{table}
  \begin{columns}
    \column{0.5\textwidth}
    \begin{enumerate}
      \item Weinmann et al\footnotemark ~estudiaron los procesos de: 
        \begin{itemize}
          \item Segmentación.
          \item Detección.
          \item Clasificación.
        \end{itemize}
    \end{enumerate}
    \column{0.5\textwidth}
    \begin{enumerate}
      \item Justifican la importancia de las características de:  
        \begin{itemize}
          \item Omnivarianza.
          \item Entropía de los valores singulares.
          \item Verticalidad del vecindario.
        \end{itemize}
    \end{enumerate}
\end{columns}
\footcitetext{3DNSSMetrics}
\end{frame}


\subsection{Modelo VQA-PC}

\begin{frame}
  \frametitle{Experimentos preliminares VQA-PC}
\begin{table}[htp]
  \small
  \begin{center}
    \begin{tabular}[c]{|c|c|c|}
      \hline
      \rowcolor[HTML]{FFC702}
      \textbf{Kfold} & \textbf{MSE} & \textbf{SROCC} \\ 
      \hline 
      0 & 13.9222 & 0.8995 \\
      \hline 
      1 & 418120.5625 & 0.8547 \\ 
      \hline 
      2 & 10.9271 & 0.9081 \\
      \hline 
      3 & 19.8226 & 0.9295 \\ 
      \hline 
      4 & 443.6077 & 0.8700 \\ 
      \hline 
      5 & 28.3165 & 0.9544 \\ 
      \hline 
      6 & 292.239 & 0.7675 \\ 
      \hline 
      7 & 329.0685 & 0.8833 \\ 
      \hline 
      8 & 357.0455 & 0.8647 \\ 
      \hline
      \textbf{\cellcolor[HTML]{FFC702}Promedio} & \textbf{46623.94} & \textbf{0.8813} \\ 
      \hline
    \end{tabular}
  \end{center}
  \caption[Resultados de experimento preliminar VQA-PC en SJTU.]{
    Resultados de experimento preliminar VQA-PC\footnotemark[13] en SJTU\footnotemark[7]. 
  }
  \label{tab:PreTestResults}
\end{table}
\footnotetext[13]{\cite{VQA-PC}}
\footnotetext[7]{\cite{SJTU}}
\end{frame}


\begin{frame}
  \frametitle{Modificaciones VQA-PC}
\begin{table}[htp]
  \small
  \centering
\begin{tabular}{|c|cccc|}
\hline
\rowcolor[HTML]{FFC702}
                       & \multicolumn{4}{c|}{\textbf{Valor medio SROCC}}                                                                                                    \\ \hline
\rowcolor[HTML]{FFC702}
\textbf{Modelo}        & \multicolumn{1}{c|}{\textbf{Estándar}} & \multicolumn{1}{c|}{\textbf{Normalizado}} & \multicolumn{1}{c|}{\textbf{Reescalado}} & \textbf{Ambos}  \\ \hline
\textbf{VQA-PC (SJTU)} & \multicolumn{1}{c|}{0.7094}            & \multicolumn{1}{c|}{0.6235}      & \multicolumn{1}{c|}{\textbf{0.8425}}    & 0.7126          \\ \hline
\end{tabular}
\caption[Valor medio sobre imágenes médicas.]{Tabla de resultados iniciales sobre imágenes médicas.}
\label{tab:SroccMedRes}
\begin{enumerate}
    \item Experimentamos con \textbf{etiquetas normalizadas o no}.
    \item En vez de recortar una selección local, \textbf{reescalar la imagen entera}.
    \item Es evidente la \textbf{importancia del reescalado}.
  \end{enumerate}
\end{table}
\end{frame}

\begin{frame}
  \frametitle{Modificaciones VQA-PC}
  \begin{enumerate}
    \item Abouelaziz et al\footnotemark~ experimentaron \textbf{distintos métodos de fusión de características}.
      \begin{itemize}
        \item Fusión por \textbf{concatenación} (F0).
        \item Fusión por \textbf{multiplicación} (F1). 
        \item Fusión por \textbf{convolución 1x1} (F2).
        \item Fusión por \textbf{\emph{compact multi-linear pooling}} (F3).
      \end{itemize}
  \end{enumerate}
  \begin{columns}
    \column{0.5\textwidth}
\begin{table}[htp] 
  \scriptsize
  \centering
  \begin{tabular}{|c|c|c|c|}
\hline
\rowcolor[HTML]{FFC702}
                       & \multicolumn{3}{c|}{\textbf{SROCC}}                                                                                                          \\ \hline
\rowcolor[HTML]{FFC702}
\textbf{Modelo}        & \multicolumn{1}{c|}{\textbf{Media}} & \multicolumn{1}{c|}{\textbf{Desviación}} & \multicolumn{1}{c|}{\textbf{Mediana}} \\ \hline
\textbf{VQA-PC F0} & \multicolumn{1}{c|}{\textbf{0.8261}}   & \multicolumn{1}{c|}{0.1589}      & \multicolumn{1}{c|}{\textbf{0.8657}}      \\ \hline
\textbf{VQA-PC F1} & \multicolumn{1}{c|}{0.8164}   & \multicolumn{1}{c|}{0.1752}      & \multicolumn{1}{c|}{0.8637}      \\ \hline
\textbf{VQA-PC F2} & \multicolumn{1}{c|}{0.8057}   & \multicolumn{1}{c|}{0.1741}      & \multicolumn{1}{c|}{0.8538}      \\ \hline
\textbf{VQA-PC F3} & \multicolumn{1}{c|}{0.7482}   & \multicolumn{1}{c|}{\textbf{0.1326}}      & \multicolumn{1}{c|}{0.7518}      \\ \hline
  \end{tabular}
  \caption[Análisis de mejoras de fusión de características en VQA-PC sin pre-entrenar.]{
    Análisis de mejoras de fusión de características en VQA-PC sin pre-entrenar.
}
\label{tab:VQAFromScratch}
\end{table}
    \column{0.5\textwidth}
\begin{table}[htp]
  \scriptsize 
  \centering
\begin{tabular}{|c|c|c|c|}
\hline
\rowcolor[HTML]{FFC702}
                       & \multicolumn{3}{c|}{\textbf{SROCC}}                                                                                                          \\ \hline
\rowcolor[HTML]{FFC702}
\textbf{Modelo}    & \textbf{Media} & \textbf{Desviación} & \textbf{Mediana} \\ \hline
\textbf{VQA-PC F0} & 0.8325           & 0.2017              & 0.9140           \\ \hline
\textbf{VQA-PC F1} & 0.8242           & 0.2025              & 0.9095           \\ \hline
\textbf{VQA-PC F2} & \textbf{0.8757}  & \textbf{0.1468}     & \textbf{0.9347}  \\ \hline
\textbf{VQA-PC F3} & 0.8071           & 0.1811              & 0.8692           \\ \hline
\end{tabular}
\caption[Análisis de mejoras de fusión de características en VQA-PC pre-entrenado.]{
  Análisis de mejoras de fusión de características en VQA-PC pre-entrenado en LS-PCQA.
}\label{tab:VQAFromLSPCQA}
\end{table}

  \end{columns}
  \vspace{-.1cm}
  \footcitetext{EnsemblePCQA}
\end{frame}
