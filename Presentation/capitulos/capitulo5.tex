\section{Conclusiones y trabajos futuros}
\begin{frame}
  \frametitle{Conclusiones}
  \begin{enumerate}
    \item \textbf{Primer método} que estima la calidad de reconstrucciones biomédicas 3D.
    \item Se logra generar un \textbf{conjunto de datos médicos sintéticos } para PCQA.
    \item Pese a ser un estudio preliminar: 
      \begin{itemize}
        \item Se justifica el uso de modelos de aprendizaje profundo experimentalmente.
        \item Obtenemos una \textbf{alta correlación (88\%)}. 
        \item Indicador de lo \textbf{prometedora} que es esta \textbf{línea de investigación}.
      \end{itemize}
    \item Se han completado satisfactoriamente los objetivos planteados.
    \item \url{https://github.com/CodeBoy-source/TFG_NRPCQA} 
  \end{enumerate}
\end{frame}

\begin{frame}
  \frametitle{Trabajos futuros}
  \begin{enumerate}
    \item Rehacer el experimento con \textbf{etiquetas generadas manualmente}. 
    \item \textbf{Para mejorar el modelo}, se podria \textbf{permitir la adaptación} del modelo de extracción de características temporales.
    \item Simular distorsiones sobre las \textbf{imágenes 2D} para obtener datos más \textbf{realistas}.
    \item Explorar \textbf{otros métodos} de la literatura. 
  \end{enumerate}
\end{frame}
