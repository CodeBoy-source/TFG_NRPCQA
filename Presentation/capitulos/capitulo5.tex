\section{Conclusiones y trabajos futuros}
\begin{frame}
  \frametitle{Conclusiones}
  \begin{enumerate}
    \item \textbf{Primer método} que estima la calidad de nubes de puntos biomédicas 3D.
    \item Se logra generar un \textbf{conjunto de datos médicos sintéticos } para PCQA.
    \item Pese a ser un estudio preliminar: 
      \begin{itemize}
        \item Se justifica el uso de modelos de aprendizaje profundo experimentalmente.
        \item Obtenemos una \textbf{alta correlación (88\%)}. 
        \item Indicador de lo \textbf{prometedora} que es esta \textbf{línea de investigación}.
      \end{itemize}
    \item Se han completado satisfactoriamente los objetivos planteados.
    \item \url{https://github.com/CodeBoy-source/TFG_NRPCQA} 
  \end{enumerate}
\end{frame}

\note{
Al recapitular un poco, hemos logrado un primer método que 
estima la calidad de nubes de puntos biomédicas. 
Donde experimentalmente, se justifica la tendencia del estado del arte al DL.
También, hemos generado un conjunto de datos médicos sintéticos. 
Y pese a ser un estudio preliminar, observamos lo prometedor que es esta 
línea de investigación.
Por lo tanto, se concluye que se han completado satisfactoriamente los objetivos 
planteados. Abriendo así, puertas a futuras investigaciones. 
Siendo un proyecto en una nueva línea de investigación, existen varias ampliaciones 
lógicas que se pueden realizar a este proyecto. 
}

\begin{frame}
  \frametitle{Trabajos futuros}
  \begin{enumerate}
    \item Rehacer el experimento con \textbf{etiquetas generadas manualmente}. 
    \item \textbf{Para mejorar el modelo}, se podria \textbf{permitir la adaptación} del modelo de extracción de características temporales.
    \item Simular distorsiones sobre las \textbf{imágenes 2D} para obtener datos más \textbf{realistas}.
    \item Explorar \textbf{otros métodos} de la literatura. 
  \end{enumerate}
\end{frame}

\note{
Se debería rehacer el experimento con un conjunto de datos etiquetado 
manualmente para volver a validar los resultados. 
Para mejorar el modelo VQA-PC, podríamos permitir la adaptación del modelo 
que extrae características del vídeo, que de momento es apenas un paso previo.
Y por último, se deberían simular las distorsiones sobre los cortes anatómicos 2D  
en vez de directamente sobre las reconstrucciones 3D para obtener distorsiones 
más realistas. 
Con todo esto, se termina mi presentación.
Muchas gracias por escuchar.
Estoy dispuesto a responder cualquier duda, pregunta o comentario.
}
